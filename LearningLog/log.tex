\documentclass[12pt]{article}
\begin{document}
\title{2ME3 Learning Log}
\maketitle

\section{January}


 \begin{tabular}{|p{2cm}| p{12cm}|} 
 \hline
 Date & Activity   \\  
 \hline
 2019-01-09 & First lecture and Attempted download of Graphviz, Make, Doxygen, Git (successful), Attempted download of Tex live on windows twice \\
 \hline
 2019-01-10 & Second lecture and Completed T01b Doxygen exercise Circle.py\\
 \hline
 2019-01-11 & Third lecture, modified docConfig\\
 \hline
 2019-01-13 &Completed some functions for ReadAllocationData.py file \\
 \hline
 2019-01-14 &Stayed with TA after tutorial to try and get doxygen/latex to work on my computer. The attempt was unsuccessful, so I have decided to focus on the assignment, and make sure everything works on mills. \\
 \hline
 2019-01-15 &Completed sort() for CalcModule.py file \\
 \hline
 2019-01-16 &Completed readStdnts() and average() for CalcModule.py file \\
 \hline
 2019-01-17 &Completed allocate() for CalcModule.py file, and updated learning log \\
 \hline
 2019-01-18 &Completed Assignment 1 Part 1. Encountered some issues related to working on both my laptop and on mills. Overall, part 1 was not too difficult, it just took a bit longer to get used to doxygen rules and learning/experimenting with testing files. Today's lecture had a big emphasis on modularity and how it can be optimized for good software programming.\\
 \hline
  2019-01-23 &Received partner files. In lecture today, a topic of importance included designing for change, and the use/design of modules for software development that has change in mind. \\
 \hline
   2019-01-24 & Found very detrimental mistakes as a result of testing partner files, relating to my not reading the assignment description properly. Likely lost a large amount of marks. This experience will encourage me to be more careful and attentive for the next assignment  \\
 \hline
   2019-01-25 & Handed in Part 2 of the assignment \\
 \hline
   2019-01-31 & Dr. Smith expressed his disappointment in our class today due to lack of participation. I have decided that from today onwards, I will do my best to read HS textbook to the best of my ability in order to be prepared to participate in class. \\
 \hline

 
\end{tabular}

\section{February}


 \begin{tabular}{|p{2cm}| p{12cm}|} 
 \hline
 Date & Activity  \\ 
 \hline
   2019-02-01 & Attended the software section of 2ME3. I was pleasantly surprised to find that many of the students in the class would regularily participate. Today, Dr.Smith talked about the difference between abstract data types and abstract objects. Despite this, I still had some confusion. I need to review this topic again later \\
 \hline
   2019-02-09 & Today I read through the entire specifications for assignment 2, and completed all modules except for Read.py and SALst.py \\
 \hline
    2019-02-10 & Ran into some errors while working on the assignment. I was confused by what an abstract object really was. I later found out that the SeqADT was a ADT, as the name describes, AllocStdntTypes was a module containing types to be used, and the Lst were Objects, as there could only be one of each \\
 \hline
   2019-02-11 & Finished the modules, and created a test file. It took a while to figure out how tests were made, and since pytest didn't work on my computer, I always had to push my files and test on mills. \\
 \hline
    2019-02-15 & Worked on the rest of the test cases \\
 \hline
    2019-02-16 & Achieved a 100 coverage on my test cases. Completed majority of report, checked avenue discussion.\\
 \hline
     2019-02-17 & Completed and edited report. Doxygen documentation in my files for A2 were also fixed. \\
 \hline
      2019-02-26 & Read over the A3 spec.pdf  \\
 \hline
       2019-02-27 & Worked on A3 spec.pdf and updated log.tex  \\
 \hline

 	
\end{tabular}

\section{March}


 \begin{tabular}{|p{2cm}| p{12cm}|} 
 \hline
 Date & Activity  \\ 
 
 \hline
        2019-03-02 & Finished A3 and updated log.tex, made notes on the past two lectures \\
 \hline
         2019-03-03 & Forgot the last question of A3, which I finished now. \\
 \hline
          2019-03-04 & Revisiting concepts from previous lectures, making more concise notes. While studying, I found that although doing assignments helped me to learn by applying concepts, studying for the midterm forced me to look at lecture content more closely, and reinforced my knowledge.  \\
 \hline
           2019-03-05 & Finished 2017 midterm  \\
 \hline
            2019-03-06 & Finished 2018 midterm, and wrote the midterm today.  \\
 \hline
             2019-03-13 & Read over the A3 Part 2 Portion  \\
 \hline
 	
\end{tabular}


\end{document}
