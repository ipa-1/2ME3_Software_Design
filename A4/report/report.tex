\documentclass[12pt]{article}

\usepackage{graphicx}
\usepackage{paralist}
\usepackage{amsfonts}
\usepackage{amsmath}
\usepackage{hhline}
\usepackage{booktabs}
\usepackage{multirow}
\usepackage{multicol}
\usepackage{url}

\oddsidemargin -10mm
\evensidemargin -10mm
\textwidth 160mm
\textheight 200mm
\renewcommand\baselinestretch{1.0}

\pagestyle {plain}
\pagenumbering{arabic}

\newcounter{stepnum}

%% Comments

\usepackage{color}

\newif\ifcomments\commentstrue

\ifcomments
\newcommand{\authornote}[3]{\textcolor{#1}{[#3 ---#2]}}
\newcommand{\todo}[1]{\textcolor{red}{[TODO: #1]}}
\else
\newcommand{\authornote}[3]{}
\newcommand{\todo}[1]{}
\fi

\newcommand{\wss}[1]{\authornote{blue}{SS}{#1}}

\title{Assignment 3, Part 1, Specification}
\author{SFWR ENG 2AA4}

\begin {document}

\maketitle
This Module Interface Specification (MIS) document contains modules, types and
methods for implementing the state of a game of Conway's Game of Life. The game involves a grid of size N x M, of cells that can be ``ALIVE'' or ``DEAD''. The player can set the start state of the game through a txt file that will be read by the program. The program will then construct the game according to the txt file allow for the user to move to the next ``move'', display the current gameboard, and write the gameboard to the same format as the read file


\newpage

\section* {Board Types Module}

\subsection*{Module}

BoardTypes

\subsection* {Uses}

N/A

\subsection* {Syntax}

\subsubsection* {Exported Constants}
SIZE = 4

\subsubsection* {Exported Types}
cellT = \{ALIVE, DEAD\} 

\subsubsection* {Exported Access Programs}

None

\subsection* {Semantics}

\subsubsection* {State Variables}

None

\subsubsection* {State Invariant}

None

\newpage

\section* {Game Board ADT Module}

\subsection*{Template Module}

BoardT

\subsection* {Uses}

\noindent BoardTypes

\subsection* {Syntax}



\subsubsection* {Exported Access Programs}

\begin{tabular}{| l | l | l | l |}
\hline
\textbf{Routine name} & \textbf{In} & \textbf{Out} & \textbf{Exceptions}\\
\hline
new BoardT  & $s: \mbox{string}$ & BoardT & \\
\hline
read & $s: \mbox{string}$ & & invalid\_argument\\
\hline
write & $s: \mbox{string}$ & & \\
\hline
getRows &  & $\mathbb{N}$ &\\
\hline
getColumns &  & $\mathbb{N}$ & \\
\hline
nextState &  & BoardT &\\
\hline
getCell& $\mathbb{N}$, $\mathbb{N}$ & cellT &invalid\_argument \\
\hline

\end{tabular}

\subsection* {Semantics}

\subsubsection* {State Variables}

$B$: seq of (seq of T) \textit{\# Gameboard}\\
$rows$: int \textit{\# Number of rows in gameboard}\\
$columns$: int \textit{\# Number of columns in gameboard}\\

\subsubsection* {State Invariant}

\subsubsection* {Assumptions \& Design Decisions}

\begin{itemize}

\item The BoardT constructor is called before any other access
  routine is called on that instance. Once a BoardT has been created, the
  constructor will not be called on it again.

\item The input file will match the given specification

\item For better scalability, this module is specified as an Abstract Data Type
  (ADT) instead of an Abstract Object. This would allow multiple games to be
  created and tracked at once by a client.

\end{itemize}

\subsubsection* {Access Routine Semantics}

\noindent BoardT($\mathit{s}$):
\begin{itemize}
\item transition: 
$B := \text{read}(s) $

\item exception: $ none$
\end{itemize}

\noindent read($s$):

\begin{itemize}
\item transition: read data from the file  associated with the string s.
  The data in the text file will be used to initiate the gameboard state variables $B$, $rows$ and $columns$, where $rows$ is the number of lines in the file, $columns$ is the number of characters on each line in the file, and $B$ holds the CellT at a particular line and position in the line. The line number and position of the character in the file must be reflected in $B$. 

  The text file has the following format, where `` `` stand for a CellT with a value of DEAD  and ``o'' stands for CellT with a value of ALIVE. All characters in the file will either be `` `` or  ``o''. lines are separated by a carriage return.  There can be any finite number of lines in the file with finite number of characters on each line as long as the length of each line in the file is the same for every line.  If the length of each line in the file is not the same for the whole file, there is an invalid\_argument exception. If the file is empty, there is an invalid\_argument exception.

\item exception: invalid\_argument
\end{itemize}

\noindent write($s$):

\begin{itemize}
\item transition: write gameboard state to the file  associated with the string s.

  The text file has the following format, where `` `` will be used for cellT with a value of DEAD and ``o'' will be used for cellT with a value of ALIVE. lines are separated by a carriage return. Each line will represent a seq in $B$. The line number and position of the character in $B$ must be reflected in the file.


\item exception: none
\end{itemize}


\noindent getRows():
\begin{itemize}
\item output: $out := rows$
\item exception: none
\end{itemize}

\noindent getColumns():
\begin{itemize}
\item output: $out := columns$
\item exception: none
\end{itemize}

\noindent getCell($n_0,n_1$):
\begin{itemize}
\item output: $out := B[n_0][n_1]$
\item exception: $exc := (\lnot \text{isValidCell}(n_0,n_1) ) \Rightarrow \text{invalid\_argument})$

\end{itemize}

\noindent nextState():
\begin{itemize}
\item output: $out := new BoardT(s, rows, columns) \text{ such that } (\forall\, i \in [0..rows-1] : (\forall\, j \in [0..columns-1] : survives(i,j) \Rightarrow s[i][j] = ALIVE \wedge  \neg survives(i,j) \Rightarrow s[i][j] = DEAD) ) $
\item exception: none
\end{itemize}



\subsection*{Local Functions}

\noindent BoardT: $ \text{(seq of seq of CellT)} \times \mathbb{N} \times \mathbb{N} \rightarrow BoardT$\\
\noindent BoardT($\mathit{s,r,c}$)
 
$ \equiv B_2 \text{ such that }  (B_2.getRows() = r \wedge B_2.getColumns = c \wedge  (\forall\, i \in [0..r - 1] : (\forall\, j \in [0..c - 1] : B_2[i][j] = s[i][j]) )
$\\


\noindent isValidCell: $\mathbb{N}  \times \mathbb{N} \rightarrow \mathbb{B} $ \\ 
\noindent isValidCell($n_0,n_1$)  $ \equiv (n_0 < 0 \lor n_0 \ge rows \lor n_1 < 0 \lor n_1 \ge columns) \Rightarrow \text{false}$ \\ 


  
\noindent neighbourCount: $\mathbb{N}  \times \mathbb{N} \rightarrow \mathbb{N}$ \\ 
\noindent neighbourCount($ n_0, n_1$) $\equiv +(\forall i,j: \mathbb{N} | i \in (n_0-1, n_0+1) \wedge
 j\in (n_1-1, n_1+1) \wedge isValidCell(i,j) \wedge (getCell(i,j)) = ALIVE : 1)$ \\


  

 \noindent survives:  $\mathbb{N}  \times \mathbb{N} \rightarrow \mathbb{B}$ \\ 
\noindent survives ($n_0, n_1$) $\equiv$\\



\begin{tabular}{|p{4cm}|p{7cm}|p{4.5cm}|}
\hhline{|-|-|-|}
$B[n_0][n_1] =  \mbox{DEAD}$ & $neighbourCount(n_0,n_1) = 3$ & True  \\
\hhline{|~|-|-|}
 & $neighbourCount(n_0,n_1) \ne 3$ & False\\
\hhline{|-|-|-|}
$B[n_0][n_1] =  \mbox{ALIVE}$ & $neighbourCount(n_0,n_1)= 3 
\lor neighbourCount(n_0,n_1) = 2 $ & True\\
\hhline{|~|-|-|}
&$neighbourCount(n_0,n_1) \neq 3 
\wedge neighbourCount(n_0,n_1) \neq 2 $ & False\\
\hhline{|-|-|-|}
\end{tabular}\\\\

\newpage

\section* {View Module}

\subsection* {Module}

View

\subsection* {Uses}

BoardTypes
GameBoard

\subsection* {Syntax}

\subsubsection* {Exported Constants}

None

\subsubsection* {Exported Access Programs}

\begin{tabular}{| l | l | l | l |}
\hline
\textbf{Routine name} & \textbf{In} & \textbf{Out} & \textbf{Exceptions}\\
\hline
new View &  & View & ~\\
\hline
print & $s: \mbox{BoardT}$ & ~ & ~\\
\hline

\end{tabular}

\subsection* {Semantics}

\subsubsection* {Environment Variables}

\subsubsection* {State Variables}

None

\subsubsection* {State Invariant}

None

\subsubsection* {Assumptions}

\begin{itemize}

\item The output file will match the given specification.

\item The boardT that is passed to the print function is initialized successfully


\end{itemize}



\subsubsection* {Access Routine Semantics}

\noindent View($s$)
\begin{itemize}
\item Constructor for view

\item exception: none
\end{itemize}

\noindent print($s$)
\begin{itemize}
\item displays the state of the gameboard using text-based ASCII graphics. 

The view has the following format, where `` `` stand for a CellT with a value of DEAD  and ``o'' stands for CellT with a value of ALIVE. All characters in the view will either be `` `` or  ``o''. The number of lines and characters on each line depends on the row and column state variables respectively, from the gameboard. The line number and position of the character in the view must be consistent with the gameboard's $B$ state variable. 

  \begin{equation}
    \begin{array}{ccc}
      o & o & o   \\
      o & o & o   \\
      o & o & o  \\

    \end{array}
  \end{equation}

\item exception: none
\end{itemize}

\newpage

\section*{Critique of Design}

This specification was written with a mix of natural language and formal language. Natural language was used for the view's print method, as there are properties in prints to the console that cannot be described by using formal language. Natural language was also used for the read and write of the gameboard module, as the format of the text file has properties that cannot be described using formal language. By using formal language for the other functions in the module, the specification that was written in formal language is unambiguous by definition. Parnas tables were used accordingly to ensure that the specification is complete, and covers all necessary cases.\\

There are two modules in this specification: One module for view, called View, and one module for the model, called BoardT. The View module displays the gameboard, and the BoardT module represents the state of the game, and has functions that allow for initializing the state, exporting the state, and calculating the state in the next  ``turn'' of the game. The components in this module are closely related, showing high cohesion, which is a desired property in software design. By having separte modules for  separate concerns: displaying and state functionality, this design shows good modularity, and separation of concerns. Although the View module uses some access programs provided by the BoardT module to access information about the gameboard, it does not rely on the functions in BoardT to display, showing a reasonable amount of coupling between the two modules.\\

The BoardT module allows for Generality, since it allows for gameboards of any length and width as long as the length of each line in the file is consistent. However, many other aspects and functions such as the character to represent a ``live cell'' have been constrained, in order to reduce possible errors and bugs. For example, if a live cell could be represented by any character in the english alphabet, in the next ``turn'' of the gameboard, there are no rules implemented in the game to determine what will represent the ``live'' and ``dead'' cells. For simplicity, only the ``o'' character can be used.\\

The modules in this design were designed to be opaque and essential, as all helper functions were made local functions, and all necessary functions for the module were made exported access programs. Each function in the BoardT module, with the help of several helper functions are minimal, each function performs a clear and specific function allowing the design to be general. For example, getRows exported access program is an accessor, that does not mutate at the same time, while having a descriptive function name. The variable names in the design were consistent. For example, for the 2D sequence, the n\_0, and is always first and always represents the ``row'' index, and the n\_0 is always second and always represents the ``column'' index.\\
\end {document}